% !TEX program = xelatex
\documentclass{VUMIFPSkursinis}
\usepackage{algorithmicx}
\usepackage{algorithm}
\usepackage{algpseudocode}
\usepackage{amsfonts}
\usepackage{amsmath}
\usepackage{bm}
\usepackage{caption}
\usepackage{color}
\usepackage{float}
\usepackage{graphicx}
\usepackage{listings}
\usepackage{subfig}
\usepackage{array}
\usepackage{wrapfig}

\usepackage[hidelinks]{hyperref}
\usepackage{multirow}
\usepackage{longtable}
\usepackage{makecell}
\usepackage{diagbox}


\newcolumntype{L}[1]{>{\raggedright\let\newline\\\arraybackslash\hspace{0pt}}p{#1}}
\newcolumntype{C}[1]{>{\centering\let\newline\\\arraybackslash\hspace{0pt}}p{#1}}
\newcolumntype{R}[1]{>{\raggedleft\let\newline\\\arraybackslash\hspace{0pt}}p{#1}}

% Titulinio aprašas
\university{Vilniaus universitetas}
\faculty{Matematikos ir informatikos fakultetas}
\department{Informatikos institutas}
\papertype{Laboratorinis darbas}
\title{Reikalavinų inžinerijos antras laboratorinis darbas}
\titleineng{Requirements engineering}
\status{1 kurso magistrantūros studentai}
\author{Šarūnas Kazimieras Buteikis}
\secondauthor{Matas Savickis}
\thirdauthor{Rokas Ulickas}
\fourthauthor{Vytautas Krivickas}
% \secondauthor{Vardonis Pavardonis} % Pridėti antrą autorių
\supervisor{dr. Audronė Lupeikienė}
\date{Vilnius – \the\year}

% Nustatymai
% \setmainfont{Palemonas} % Pakeisti teksto šriftą į Palemonas (turi būti įdiegtas sistemoje)
\bibliography{bibliografija}

\begin{document}

\maketitle

\sectionnonumnocontent{Santrauka}

Šiame dokumente pateikiamas „Epidemiologinės šalies situacijos sekimo sistemos“ reikalavimų specifikacijos validavimas ir verifikavimas, atliktas
perspektyva paremtu skaitymu bei reikalavimų nuleidimas žemyn -- kokybės namai. Komandą sudarė (pateikiamos pasirinktos perspektyvos):
\begin{itemize}
	\item Šarūnas Kazimieras Buteikis (el. paštas \href{mailto:sarunas.buteikis@mif.stud.vu.lt}{sarunas.buteikis@mif.stud.vu.lt}) -- vartotojo perspektyva.
	\item Vytautas Krivickas (el. paštas \href{mailto:vytautas.krivickas@mif.stud.vu.lt}{vytautas.krivickas@mif.stud.vu.lt}) -- .
	\item Matas Savickis (el. paštas \href{mailto:matas.savickis@mif.stud.vu.lt}{matas.savickis@mif.stud.vu.lt}) -- .
	\item Rokas Ulickas (el. paštas \href{mailto:rokas.ulickas@mif.stud.vu.lt}{rokas.ulickas@mif.stud.vu.lt}) -- .
\end{itemize}

\newpage

\tableofcontents


\section{Įžanga}
Šiame dokumente aprašoma „Epidemiologinės šalies situacijos sekimo sistemos”, toliau - „epidemiologinė sistemos” arba „sistemos”
reikalavimų validacija ir verifikacija bei reikalavimų nuleidimas žemyn.
Ši sistema skirta sekti epidemiologinei padėčiai šalyje: įvertinti viruso plitimo šalyje tendencijas,
efektyviai identifikuoti naujus viruso židinius, leisti specialistams atsekti susirgusiųjų
kontaktus registruojant užsikrėtusiųjų maršrutus ir potencialiuose rizikos židiniuose
besilankančius žmones, greitai informuoti kontaktavusiuosius su užsikrėtusiu žmogumi
apie privalomą saviizoliaciją, rinkti duomenis apie asmenis karantine.

\subsection{Pritaikymo sritis}
Ši sistema skirta naudoti sveikatos apsaugos sistemoje: sistema turėtų palengvinti
epidemiologų darbą ir leisti sekti viruso plitimą populiacijoje, imtis efektyvesnės
profilaktikos ir tirti naudojamų priemonių efektyvumą.

\subsection{Probleminė sritis}
Sistema siekiama išspręsti šias problemas:
\begin{itemize}
	\item Atskirų sveikatos įstaigų renkami susirgimų duomenys nėra apdorojami centralizuotai
	      arba tai daroma ne sistemingai, todėl epidemiologams sunku identifikuoti tikrąsias viruso
	      plitimo šalyje tendencijas, greitai identifikuoti potencialius židinius.
	\item Dėl žmogiškųjų resursų trūkumo dažnai tampa neįmanoma įspėti visų kontaktavusiųjų
	      su užsikrėtusiuoju asmenų - automatizavus šį procesą būtų galima įgyvendinti efektyvesnę
	      profilaktiką, užkardyti nevaldomą epidemijos plitimą.
	\item Šiuo metu nėra centralizuotos sistemos, leidžiančios registruoti potencialiuose
	      rizikos židiniuose (įvairiuose renginiuose, masinio susibūrimo vietose) besilankančius
	      asmenis, dabar egzistuojančios pavienės iniciatyvos neleidžia automatiškai atsekti reikšmingo kiekio susirgusiojo kontaktų - tenka pasikliauti pastarojo pateikta informacija.
	\item Nacionalinio sveikatos centro darbuotojai neturi galimybės automatiškai įspėti
	      atvykusiųjų iš pavojingų šalių asmenų apie privalomą saviizoliaciją: atlikus reikiamas
	      integracijas su muitinės sistemomis, ši sistema leistų automatizuoti ir šį procesą.
	\item Šiuo metu nėra galimybės automatizuoti saviizoliacijos reikalavimų laikymosi sekimo,
	      tad naujoji sistema leistų bent iš dalies automatizuoti šį procesą: reikalauti asmenis
	      saviizoliacijoje pateikti savo dabartinę vietą naudojantis išmaniajame telefone esančia
	      GPS sistema ar atsiųsti saviizoliaciją patvirtinančią nuotrauką.
\end{itemize}

\subsection{Naudotojai}
Šios sistemos naudotojų bazę sudaro trijų kategorijų naudotojai:
\begin{itemize}
	\item Epidemiologai - tai savo srities ekspertai, turintys aukštąjį išsilavinimą.
	      Naudotis sistema jiems pakaks mokykloje dėstomo informatikos kurso.
	\item LR esantys asmenys, dalyvaujantys riziką turinčiuose renginiuose, esantys saviizoliacijoje,
	      atvykę iš pavojingų šalių ar turėję sąlytį su sergančiaisiais - jiems taip pat pakaks
	      mokykloje dėstomo informatikos kurso.
	\item Duomenų analitikai - tam, jog galėtų efektyviai panaudoti sistemoje
	      esančius duomenis jiems reikalingas bakalauro ar aukštesnis iššsilavinimas
	      duomenų mokslo ar informatikos srityse.
\end{itemize}

\section{Reikalavimų validacija and verifikacija}

Šiame skyriuje aprašoma reikalavimų validacija ir verifikacija: siekiama rasti dokumentuotuose programinės įrangos
reikalavimuose siekiama identifikuoti klaidas, tokias kaip dviprasmiškumas, neužbaigtumas, prieštaringumas ir kt.
Šiam tikslui įgyvendinti naudojamas perspektyva paremtas skaitymas (angl. \textit{perspective-based reading}). Komandos narių
pasirinktos perspektyvos -- rolės -- pateikiamos santraukoje.

\subsection{Klientas}

\subsubsection{Programinės įrangos reikalavimų specifikacijų vertinimo kontrolinis sąrašas}

\subsubsection{Programinės įrangos sistemos reikalavimų specifikacijos įvertinimas}

\subsubsection{Pataisyta programinės įrangos reikalavimų specifikacijos versija}

		\subsection{Vartotojas}
			\subsubsection{Programinės įrangos reikalavimų specifikacijų vertinimo kontrolinis sąrašas}
			\begin{center}
				\setstretch{1.0}

				\begin{longtable}{|C{2cm}|L{5cm}|L{8cm}|}

					\caption{Vartotojo kontrolinis sąrašas}
					\label{table:VKS}

 					\\ \hline
 					\multicolumn{1}{|c|}{\makecell{\textbf{Kodas}}} &
  					\multicolumn{1}{c|}{\makecell{\textbf{Klausimas/Teiginys}}} & 
  					\multicolumn{1}{c|}{\makecell{\textbf{Apibūdinimas}}}
 					\\ \hline
 					VKS-01 &
 					Reikalavimas parašytas vartotojui suprantama kalba - lietuvių kalba& 
 					Reikalavimas parašytas lietuvių kalba. Kadangi sistema skirta Lietuvos Respublikos gyventojams. Tikėtina, kad vartotojui suprantama kalba yra lietuvių kalba\\ \hline
 					VKS-02 &
 					Reikalavimai apibūdina, jog vartotojo sąsaja bus vartotojui suprantama kalba - lietuvių kalba&
 					Reikalavimai (ar bendras reikalavimas), nurodantis, jog vartotojo sąsajoje esantis tekstas bus pateiktas lietuvių kalba.\\ \hline
 					VKS-03 &
 					Ar reikalavimas apibūdina sistemos išorinį elgesį? & 
					Reikalavimai (ar bendras reikalavimas) apibūdina sistemos elgesį iš vartotojo perspektyvos - vartotojas paduoda specifines įvestis ir sistema gražina konkrečias išvestis.\\ \hline  
 					VKS-04 &
 					Ar reikalavimas apibrėžia, kaip sistemos vartotojo sąsaja reaguos į vartotojo interakcijas?& 
 					Reikalavimai (ar bendras reikalavimas) apibūdina, kaip sistemos vartotojo sąsaja reaguoja į vartotojo interakcijas\\ \hline    
 					VKS-05 &
 					Ar galima valdyti asmens saviizoliaciją?& 
 					Reikalavimai, apibrėžiantys, jog galima pačiam vartotojui užfiksuoti jų saviizoliaciją, įspėti vartotoją apie privalomą saviizoliaciją, nustatyti vartotojo saviizoliacijos pradžią ir pabaigą.\\ \hline  
 					VKS-06 &
 					Ar galima privačiam asmeniui būti įspėtam apie privalomą saviizoliaciją?& 
 					Reikalavimai, apibrėžiantys, jog vartotojui pranešama apie privalomą saviizoliaciją\\ \hline   					
 					VKS-07 &
 					Ar galima privačiam asmeniui užfiksuoti kontaktuotus žmones?& 
 					Reikalavimai, apibrėžiantys, jog vartotojas gali užfiksuoti asmenis, kurie kontaktavo su užsikrėtusiuoju\\ \hline  
 					VKS-08 &
 					Ar galima sveikatos apsaugos ministerijos atstovui valdyti pavojingų šalių sąrašą?& 
 					Reikalavimai, apibrėžiantys, jog sveikatos apsaugos ministerijos atstovas gali modifikuoti pavojingų šalių sąrašą\\ \hline  
 					VKS-09 &
 					Ar gali E. policija sužinoti apie saviizoliacijos pažeidimus?& 
 					Reikalavimai, apibrėžiantys, jog E. policijai pranešama apie asmens saviizoliacijos pažeidimą\\ \hline			
				\end{longtable}
			\end{center}			

\subsubsection{Programinės įrangos sistemos reikalavimų specifikacijos įvertinimas}
			\textbf{TODO: konkretizuoti, kokie reikalavimai pažeidžiami.}
			\begin{center}
				\setstretch{1.0}

\begin{longtable}{|C{2cm}|L{2.5cm}|L{10cm}|}

					\caption{Vartotojo kontrolinis sąrašas}
					\label{table:VKS}

 					\\ \hline
 					\multicolumn{1}{|c|}{\makecell{\textbf{Kodas}}} &
  					\multicolumn{1}{c|}{\makecell{Ar\\tenkina\\reikalavimus?}} & 
  					\multicolumn{1}{c|}{\makecell{\textbf{Reikalavimų pažeidimas}}}
 					\\ \hline
 					VKS-01 &
 					Taip& 
 					--\\ \hline
 					VKS-02 &
 					Ne&
 					Nėra apibrėžtų reikalavimų ar bendro reikalavimo, nurodančio, jog vartotojo sąsajoje esantis tekstas bus lietuvių kalba\\ \hline
 					VKS-03 &
 					Taip & 
					--\\ \hline  
 					VKS-04 &
 					Ne& 
 					Reikalavimai per daug abstraktūs.\\ \hline    
 					VKS-05 &
 					Ne& 
 					Reikalavimai per daug abstraktūs. PS reikalavimuose nėra išreikštinai išskirta privačių asmenų vartotojų grupė\\ \hline  
 					VKS-06 &
 					Ne&
 					Reikalavimai per daug abstraktūs. PS reikalavimuose nėra išreikštinai išskirta privačių asmenų vartotojų grupė\\ \hline   					
 					VKS-07 &
 					Ne&
 					Reikalavimai per daug abstraktūs. PS reikalavimuose nėra išreikšinai išskirta privačių asmenų vartotojų grupė\\ \hline  
 					VKS-08 &
 					Ne&
 					Reikalavimai per daug abstraktūs. PS reikalavimuose nėra išreikštinai išskirta sveikatos apsgaugos ministerijos vartotojų grupė\\ \hline  
 					VKS-09 &
 					Ne&
 					Reikalavimai per daug abstraktūs. PS reikalavimuose nėra išreikšinai išskirta e. policijos vartotojų grupė\\ \hline			
				\end{longtable}
			\end{center}		
			
\subsubsection{Pataisyta programinės įrangos reikalavimų specifikacijos versija}
\subsection{Programuotojo perspektyva}

Šiame poskyryje pateikiama programuotojo -- asmens, kursiančio aprašomą sistemą -- perspektyva.

\subsubsection{Programinės įrangos reikalavimų specifikacijų vertinimo kontrolinis sąrašas}

Šiame skirsnyje pateikiamas programinės įrangos reikalavimų specifikacijų vertinimo kontrolinis sąrašas,
paremtas programuotojo perspektyva.

\begin{center}
	\setstretch{1.0}
	\small
	\begin{longtable}{|L{3cm}|C{5cm}|C{7cm}|}
		\caption{Programuotojo kontrolinis sąrašas}
		\label{table:EmployeeSalary}
		\\ \hline
		Klausimo identifikatorius &
		Klausimas                 &
		Apibūdinimas                \\ \hline
	\end{longtable}
\end{center}

\subsubsection{Programinės įrangos sistemos reikalavimų specifikacijos įvertinimas}
\subsubsection{Pataisyta programinės įrangos reikalavimų specifikacijos versija}
\subsection{Operacijų ir palaikymo grupė}
\subsubsection{Programinės įrangos reikalavimų specifikacijų vertinimo kontrolinis sąrašas}
\subsubsection{Programinės įrangos sistemos reikalavimų specifikacijos įvertinimas}
\subsubsection{Pataisyta programinės įrangos reikalavimų specifikacijos versija}

\section{Reikalavimų nuleidimas žemyn - kokybės namas}
\subsection{Klientas}
\subsubsection{Produkto (sistemos) planavimas}
\subsubsection{Komponentų diegimas}
\subsubsection{Komponentų diegimas}
\subsubsection{Gamybos planavimas}
\subsection{Vartotojas}
\subsubsection{Produkto (sistemos) planavimas}
\subsubsection{Komponentų diegimas}
\subsubsection{Komponentų diegimas}
\subsubsection{Gamybos planavimas}
\subsection{Programuotojas}
\subsubsection{Produkto (sistemos) planavimas}
\subsubsection{Komponentų diegimas}
\subsubsection{Komponentų diegimas}
\subsubsection{Gamybos planavimas}
\subsection{Operacijų ir palaikymo grupė}
\subsubsection{Produkto (sistemos) planavimas}
\subsubsection{Komponentų diegimas}
\subsubsection{Komponentų diegimas}
\subsubsection{Gamybos planavimas}

\section{Išvada}

\end{document}

