% !TEX program = xelatex
\documentclass{VUMIFPSkursinis}
\usepackage{algorithmicx}
\usepackage{algorithm}
\usepackage{algpseudocode}
\usepackage{amsfonts}
\usepackage{amsmath}
\usepackage{bm}
\usepackage{caption}
\usepackage{color}
\usepackage{float}
\usepackage{graphicx}
\usepackage{listings}
\usepackage{subfig}
\usepackage{array}
\usepackage{wrapfig}

\usepackage{enumitem}
\usepackage[dvipsnames,table]{xcolor}
\usepackage[hidelinks]{hyperref}
\usepackage{multirow}
\usepackage{longtable}
\usepackage{makecell}


\newcolumntype{L}[1]{>{\raggedright\let\newline\\\arraybackslash\hspace{0pt}}p{#1}}
\newcolumntype{C}[1]{>{\centering\let\newline\\\arraybackslash\hspace{0pt}}p{#1}}
\newcolumntype{R}[1]{>{\raggedleft\let\newline\\\arraybackslash\hspace{0pt}}p{#1}}

\newcounter{counter}
\newcommand{\reqCode}{%
	 \stepcounter{counter}%
	\color{blue} REQ-\thecounter}

% Titulinio aprašas
\university{Vilniaus universitetas}
\faculty{Matematikos ir informatikos fakultetas}
\department{Informatikos institutas}
\papertype{Laboratorinis darbas}
\title{Reikalavinų inžinerijos antras laboratorinis darbas}
\titleineng{Requirements engineering}
\status{1 kurso magistrantūros studentai}
\author{Šarūnas Kazimieras Buteikis}
\secondauthor{Matas Savickis}
\thirdauthor{Rokas Ulickas}
\fourthauthor{Vytautas Krivickas}
\supervisor{dr. Audronė Lupeikienė}
\date{Vilnius – \the\year}

\bibliography{bibliografija}

\begin{document}

\maketitle

\sectionnonumnocontent{Santrauka}

Šiame dokumente pateikiamas „Epidemiologinės šalies situacijos sekimo sistemos“ reikalavimų specifikacijos validavimas ir verifikavimas, atliktas
perspektyva paremtu skaitymu bei reikalavimų nuleidimas žemyn -- kokybės namai. Komandą sudarė (pateikiamos pasirinktos perspektyvos):
\begin{itemize}
	\item Šarūnas Kazimieras Buteikis (el. paštas \href{mailto:sarunas.buteikis@mif.stud.vu.lt}{sarunas.buteikis@mif.stud.vu.lt}) -- vartotojo perspektyva.
	\item Vytautas Krivickas (el. paštas \href{mailto:vytautas.krivickas@mif.stud.vu.lt}{vytautas.krivickas@mif.stud.vu.lt}) -- programuotojo perspektyva.
	\item Matas Savickis (el. paštas \href{mailto:matas.savickis@mif.stud.vu.lt}{matas.savickis@mif.stud.vu.lt}) -- .
	\item Rokas Ulickas (el. paštas \href{mailto:rokas.ulickas@mif.stud.vu.lt}{rokas.ulickas@mif.stud.vu.lt}) -- kliento perspektyva.
\end{itemize}

\newpage

\tableofcontents


\section{Įžanga}

Šiame dokumente aprašoma „Epidemiologinės šalies situacijos sekimo sistemos”, toliau - „epidemiologinės sistemos” arba „sistemos”
reikalavimų validacija ir verifikacija bei reikalavimų nuleidimas žemyn.
Ši sistema skirta sekti epidemiologinei padėčiai šalyje: įvertinti viruso plitimo šalyje tendencijas,
efektyviai identifikuoti naujus viruso židinius, leisti specialistams atsekti susirgusiųjų
kontaktus registruojant užsikrėtusiųjų maršrutus ir potencialiuose rizikos židiniuose
besilankančius žmones, greitai informuoti kontaktavusiuosius su užsikrėtusiu žmogumi
apie privalomą saviizoliaciją, rinkti duomenis apie asmenis karantine.

\subsection{Pritaikymo sritis}
Ši sistema skirta naudoti sveikatos apsaugos sistemoje: sistema turėtų palengvinti
epidemiologų darbą ir leisti sekti viruso plitimą populiacijoje, imtis efektyvesnės
profilaktikos ir tirti naudojamų priemonių efektyvumą.

\subsection{Probleminė sritis}
Sistema siekiama išspręsti šias problemas:
\begin{itemize}
	\item Atskirų sveikatos įstaigų renkami susirgimų duomenys nėra apdorojami centralizuotai
	      arba tai daroma ne sistemingai, todėl epidemiologams sunku identifikuoti tikrąsias viruso
	      plitimo šalyje tendencijas, greitai identifikuoti potencialius židinius.
	\item Dėl žmogiškųjų resursų trūkumo dažnai tampa neįmanoma įspėti visų kontaktavusiųjų
	      su užsikrėtusiuoju asmenų - automatizavus šį procesą būtų galima įgyvendinti efektyvesnę
	      profilaktiką, užkardyti nevaldomą epidemijos plitimą.
	\item Šiuo metu nėra centralizuotos sistemos, leidžiančios registruoti potencialiuose
	      rizikos židiniuose (įvairiuose renginiuose, masinio susibūrimo vietose) besilankančius
	      asmenis, dabar egzistuojančios pavienės iniciatyvos neleidžia automatiškai atsekti reikšmingo kiekio susirgusiojo kontaktų - tenka pasikliauti pastarojo pateikta informacija.
	\item Nacionalinio sveikatos centro darbuotojai neturi galimybės automatiškai įspėti
	      atvykusiųjų iš pavojingų šalių asmenų apie privalomą saviizoliaciją: atlikus reikiamas
	      integracijas su muitinės sistemomis, ši sistema leistų automatizuoti ir šį procesą.
	\item Šiuo metu nėra galimybės automatizuoti saviizoliacijos reikalavimų laikymosi sekimo,
	      tad naujoji sistema leistų bent iš dalies automatizuoti šį procesą: reikalauti asmenis
	      saviizoliacijoje pateikti savo dabartinę vietą naudojantis išmaniajame telefone esančia
	      GPS sistema ar atsiųsti saviizoliaciją patvirtinančią nuotrauką.
\end{itemize}

\subsection{Naudotojai}
Šios sistemos naudotojų bazę sudaro trijų kategorijų naudotojai:
\begin{itemize}
	\item Epidemiologai - tai savo srities ekspertai, turintys aukštąjį išsilavinimą.
	      Naudotis sistema jiems pakaks mokykloje dėstomo informatikos kurso.
	\item LR esantys asmenys, dalyvaujantys riziką turinčiuose renginiuose, esantys saviizoliacijoje,
	      atvykę iš pavojingų šalių ar turėję sąlytį su sergančiaisiais - jiems taip pat pakaks
	      mokykloje dėstomo informatikos kurso.
	\item Duomenų analitikai - tam, jog galėtų efektyviai panaudoti sistemoje
	      esančius duomenis jiems reikalingas bakalauro ar aukštesnis iššsilavinimas
	      duomenų mokslo ar informatikos srityse.
\end{itemize}

% --------------------------------------------------------------------------------------------------------------------------------------------------------------------------------
% --------------------------------------------------------------------------------------------------------------------------------------------------------------------------------
% --------------------------------------------------------------------------------------------------------------------------------------------------------------------------------

\section{Reikalavimų validacija ir verifikacija}

Šiame skyriuje aprašoma reikalavimų validacija ir verifikacija: siekiama rasti dokumentuotuose programinės įrangos
reikalavimuose siekiama identifikuoti klaidas, tokias kaip dviprasmiškumas, neužbaigtumas, prieštaringumas ir kt.
Šiam tikslui įgyvendinti naudojamas perspektyva paremtas skaitymas (angl. \textit{perspective-based reading}). Komandos narių
pasirinktos perspektyvos -- rolės -- pateikiamos santraukoje.

% --------------------------------------------------------------------------------------------------------------------------------------------------------------------------------
% --------------------------------------------------------------------------------------------------------------------------------------------------------------------------------
% --------------------------------------------------------------------------------------------------------------------------------------------------------------------------------

\subsection{Kliento perspektyva}

Šiame poskyryje pateikiama kliento -- asmens, perkančio kuriamą sistemą -- perspektyva.

\subsubsection{Programinės įrangos reikalavimų specifikacijų vertinimo kontrolinis sąrašas}

Šiame skirsnyje pateikiamas programinės įrangos reikalavimų specifikacijų vertinimo kontrolinis sąrašas, paremtas pagal kliento perspektyvą.

\begin{center}
	\setstretch{1.0}

	\begin{longtable}{|C{2cm}|L{5cm}|L{8cm}|}

		\caption{Kliento kontrolinis sąrašas}
		\label{table:KKS}

		\\ \hline
		\multicolumn{1}{|c|}{\makecell{\textbf{Kodas}}}                                               &
		\multicolumn{1}{c|}{\makecell{\textbf{Klausimas/Teiginys}}}                                   &
		\multicolumn{1}{c|}{\makecell{\textbf{Apibūdinimas}}}
		\\ \hline
		KKS-01                                                                                        &
		Ar neprieštarauja galiojantiems teisės aktams?                          					  &
		Reikalavimas atitinka teisės aktus: Lietuvos Respublikos, Europos Sąjungos bei kitus.                                     
		\\ \hline
		KKS-02                                                                                        &
		Ar reikalavimas yra tikslingas šiam projektui? 												  &
		Reikalavimas būti apibrėžtas ir tikslingas šiam projektui ir reikalavimams, kurie turi būti įgyvendinti šio ir tik šio projekto metu.                                                                      
		\\ \hline
		KKS-03                                                                                        &
		Ar reikalavimas atsižvelgia į visas vartotojų/naudotojų grupes?                               &
		Sistema turi būti patogu naudotis ir jaunam, ir senam, ir nepatyrusiam ir patyrusiam kompiuterio naudotojui.
		\\ \hline
		KKS-04                                                                                        &
		Ar reikalavimas naudingas arba neša finansinę/socialinę/ekonominę naudą?  									  &
		Kiekvienas reikalavimas turi būti naudingas būsimai sistemai ir turi turėti prasmę/naudą/tęstinumą. 
		\\ \hline 
		KKS-05                                                                                        &
		Ar reikalavimo įvykdymo kriterijai yra apibrėžti?                                             &
		Ar reikalavimas turi suformuluotus priėmimo/įvykdymo kriterijus? 
		\\ \hline
		KKS-06                                                                                        &
		Ar reikalavimas neapribotas tik vienai infekcinei ligai valdyti/sekti?                        &
		Sistema turi veikti ir turi būti galima sekti kelias infekcines ligas vienu metu, pvz.: COVID-19 ir gripą, ir tuberkuliozę.
		\\ \hline
		KKS-07                                                                                        &
		Ar reikalavimas atkreipia dėmesį į skirtingą informacijos prieigos lygį, skirtingoms rolėms?  &
		Turi būti apibrėžta, kur, kam ir kokie duomenys bus matomi sistemoje                                                                                   
		\\ \hline
		KKS-08                                                                                        &
		Ar reikalavimas pasiekiamas?            &
		Vadovaujantis sveiku protu, ar reikalavimas yra pasiekiamas?                                                                              
		\\ \hline                                                                              
	\end{longtable}
\end{center}


\subsubsection{Programinės įrangos sistemos reikalavimų specifikacijos įvertinimas}


Šiame skirsnyje pateikiamas programinės įrangos reikalavimų specifikacijų įvertinimas pagal kliento perspektyvos kontrolinį sąrašą.
\begin{center}
	\setstretch{1.0}

	\begin{longtable}{|C{2cm}|L{2.5cm}|L{10cm}|}

		\caption{Reikalavimų specifikacijos įvertinimas pagal kliento perspektyvos kontrolinį sąrašą}
		\label{table:KKS}

		\\ \hline
		\multicolumn{1}{|c|}{\makecell{\textbf{Kodas}}} &
		\multicolumn{1}{c|}{\makecell{Ar                                                                                                                                                                                                                                              \\tenkina\\reikalavimus?}} &
		\multicolumn{1}{c|}{\makecell{\textbf{Reikalavimų pažeidimas}}}
		\\ \hline
		KKS-01                                          &
		Taip                                            &
		\multicolumn{1}{|c|}{--}                                                                                                                                                                                                                                                      
		\\ \hline
		
		KKS-02                                          &
		Taip                                            &
		\multicolumn{1}{|c|}{--}                                                                                                                                                                                                                                                      
		\\ \hline

		KKS-03                                          &
		Ne                                            &
		Visi reikalavimai parašyti taip, jog neatsižvelgiame nei į vieną vartotojo tipą/rolę, nėra patikslinama su kokiomis teisėmis vartotojai gali pasiekti tam tikrus veiksmus
		\\ \hline

		KKS-04                                          &
		Taip                                              &
		\multicolumn{1}{|c|}{--}                                                                         
		\\ \hline

		KKS-05                                          &
		Ne                                              &
		Visi reikalavimai parašyti taip, jog įvykdymo kriterijai nėra paminėti.
		\\ \hline

		KKS-06                                          &
		Taip                                              &
		\multicolumn{1}{|c|}{--}  
		\\ \hline

		KKS-07                                          &
		Ne                                              &
		Visi reikalavimai parašyti taip, jog dėmesys į skirtingas informacijos prieigos lygius skirtingoms rolėms nėra atkreipiamas.
		\\ \hline

		KKS-08                                          &
		Ne                                              &
		Visi reikalavimai parašyti taip, jog nėra aišku ir nėra galima įvertinti ar reikalavimas gali būti pasiektas ar ne, kadangi reikalavimas nėra tinkamai išplėtotas
		\\ \hline
\end{longtable}
\end{center}

% --------------------------------------------------------------------------------------------------------------------------------------------------------------------------------
% --------------------------------------------------------------------------------------------------------------------------------------------------------------------------------
% --------------------------------------------------------------------------------------------------------------------------------------------------------------------------------


\subsection{Vartotojo perspektyva}

Šiame poskyryje pateikiama vartotojo -- asmens, naudojančio aprašomą sistemą -- perspektyva.

\subsubsection{Programinės įrangos reikalavimų specifikacijų vertinimo kontrolinis sąrašas}

Šiame skirsnyje pateikiamas programinės įrangos reikalavimų specifikacijų vertinimo kontrolinis sąrašas,
paremtas programuotojo perspektyva.

\begin{center}
	\setstretch{1.0}
	\small
	\begin{longtable}{|C{2cm}|L{5cm}|L{8cm}|}
		\caption{Programuotojo kontrolinis sąrašas}
		\label{table:EmployeeSalary}
		\\ \hline
		\multicolumn{1}{|c|}{\makecell{\textbf{Kodas}}}                               &
		\multicolumn{1}{c|}{\makecell{\textbf{Klausimas/Teiginys}}}                   &
		\multicolumn{1}{c|}{\makecell{\textbf{Apibūdinimas}}}
		\\ \hline
		PKS-01                                                                        &
		Ar reikalavimas abstraktus?                                                   &
		Reikalavimas vadinamas abstrak2iu, jei jis suformuluotas remiantis juodos dėžės (angl. \textit{black box}) principu, t. y. apibrėžiama tik išoriškai matoma funkcinė ar nefunkcinė sistemos savybė ir nėrapateikiama jokių tos savybės realizavimo detalių.                                                                                                                                                                                   \\ \hline
		PKS-02                                                                        &
		Ar reikalavimas atominis?                                                     &
		Reikalavimas vadinamas atominiu, jei jis nebegali būti išskaidytas į smulkesnius. Reikalavimas turėtų būti sudarytas iš vieno atsekamo (angl \textit{traceable}) elemento.                                                                                                                                                                                                                                                                    \\ \hline
		PKS-03                                                                        &
		Ar reikalavimas nėra perteklinis?                                             &
		Reikalavimas vadinamas pertekliniu (angl. \textit{redundant}), jei jame kartojama informacija, pateikiama kituose reikalavimuose arba jei neįmanoma įvardinti, kokie veslo reikalavimai bus pasiekti įgyvendinus šį reikalavimą.                                                                                                                                                                                                              \\ \hline
		PKS-04                                                                        &
		Ar reikalavimas nėra dviprasmiškas?                                           &
		Reikalavimas turėtų būti interpretuojamas \textbf{vienareikšmiškai}. Tiesa, tai nereiškia, kad jis gali būti realizuojamas tik vienu būdu.                                                                                                                                                                                                                                                                                                    \\ \hline
		PKS-05                                                                        &
		Ar reikalavimas gali būti ištestuotas?                                        &
		Testuotojams turi būti įmanoma įvertinti, ar reikalavimas įgyvendintas teisingai: testo atsakymas turi būti arba teigiamas, arba neigiamas. Tam, jog galėtų būti ištestuojamas, reikalavimai turi būti aiškūs, tikslūs ir nedviprasmiški, o verifikacijos metodas -- realistiškas, t. y., jis turėtų nebūti per brangus, užimti per daug laiko ar reikalauti specifinių tikrintojo žinių ar specialios kompiuterinės bei programinės įrangos. \\ \hline
		PKS-06                                                                        &
		Ar reikalavimas pilnas?                                                       &
		Reikalavimas yra pilnas (angl. \textit{complete}), jei jame apibrėžiama viskas, kas turi būti apibrėžta ir jį perskaičius neišryškėja trūkumų (angl. \textit{issues}). Reikalavimas turi būti pilnas vertinant ne tik visumą, bet ir reikalavimą atskirai. Reikalavimas turi apimti visas įmanomas sąlygas.                                                                                                                                   \\ \hline
		PKS-07                                                                        &
		Ar reikalavimas tikslus?                                                      &
		Reikalavimas laikomas tiksliu, jei visi jame naudojami terminai yra apibrėžti ir nėra vartojama netikslių terminų (pavyzdžiui, beveik, apytiksliai, patogu, naudojama mažai atminties ir pan.).                                                                                                                                                                                                                                               \\ \hline
		PKS-08                                                                        &
		Ar reikalavimas aiškus, neperkrautas?                                         &
		Reikalavimas vadinamas neperkrautu, jei jame nėra argumentacijos, apibrėžimų ar kitos nereikalingos informacijos.                                                                                                                                                                                                                                                                                                                             \\ \hline
		PKS-09                                                                        &
		Ar reikalavimas parašytas visiems suinteresuotiems asmenims suprantama kalba? &
		Reikalavimas laikomas suprantamu, jei jame nėra specifinių terminų, neapibrėžtų terminų žodyne ir jame aiškiai apibrėžta, kokių funkcinių ar nefunkcinių charakteristikų sistema privalo turėti. Reikalavimai turi būti teisingi gramatiškai ir parašyti laikantis vieningo stiliaus. Turi būti paisoma standartinių konvencijų.                                                                                                              \\ \hline
	\end{longtable}
\end{center}

\subsubsection{Programinės įrangos sistemos reikalavimų specifikacijos įvertinimas}

Šiame skirsnyje pateikiamas programinės įrangos reikalavimų specifikacijų vertinimo kontrolinis sąrašas, paremtas pagal vartotojo perspektyvą.

\begin{center}
	\setstretch{1.0}

	\begin{longtable}{|C{2cm}|L{5cm}|L{8cm}|}

		\caption{Vartotojo kontrolinis sąrašas}
		\label{table:VKS}

		\\ \hline
		\multicolumn{1}{|c|}{\makecell{\textbf{Kodas}}}                                               &
		\multicolumn{1}{c|}{\makecell{\textbf{Klausimas/Teiginys}}}                                   &
		\multicolumn{1}{c|}{\makecell{\textbf{Apibūdinimas}}}
		\\ \hline
		VKS-01                                                                                        &
		Reikalavimas parašytas vartotojui suprantama kalba - lietuvių kalba                           &
		Reikalavimas parašytas lietuvių kalba. Kadangi sistema skirta Lietuvos Respublikos gyventojams. Tikėtina, kad vartotojui suprantama kalba yra lietuvių kalba                                    \\ \hline
		VKS-02                                                                                        &
		Reikalavimai apibūdina, jog vartotojo sąsaja bus vartotojui suprantama kalba - lietuvių kalba &
		Reikalavimai (ar bendras reikalavimas), nurodantis, jog vartotojo sąsajoje esantis tekstas bus pateiktas lietuvių kalba.                                                                        \\ \hline
		VKS-03                                                                                        &
		Ar reikalavimas apibūdina sistemos išorinį elgesį?                                            &
		Reikalavimai (ar bendras reikalavimas) apibūdina sistemos elgesį iš vartotojo perspektyvos - vartotojas paduoda specifines įvestis ir sistema gražina konkrečias išvestis.                      \\ \hline
		VKS-04                                                                                        &
		Ar reikalavimas apibrėžia, kaip sistemos vartotojo sąsaja reaguos į vartotojo interakcijas?   &
		Reikalavimai (ar bendras reikalavimas) apibūdina, kaip sistemos vartotojo sąsaja reaguoja į vartotojo interakcijas                                                                              \\ \hline
		VKS-05                                                                                        &
		Ar galima valdyti asmens saviizoliaciją?                                                      &
		Reikalavimai, apibrėžiantys, jog galima pačiam vartotojui užfiksuoti jų saviizoliaciją, nustatyti vartotojo saviizoliacijos pradžią ir pabaigą. \\ \hline
		VKS-06                                                                                        &
		Ar galima privačiam asmeniui būti įspėtam apie privalomą saviizoliaciją?                      &
		Reikalavimai, apibrėžiantys, jog vartotojui pranešama apie privalomą saviizoliaciją                                                                                                             \\ \hline
		VKS-07                                                                                        &
		Ar galima privačiam asmeniui užfiksuoti kontaktuotus žmones?                                  &
		Reikalavimai, apibrėžiantys, jog vartotojas gali užfiksuoti asmenis, kurie kontaktavo su užsikrėtusiuoju                                                                                        \\ \hline
		VKS-08                                                                                        &
		Ar galima sveikatos apsaugos ministerijos atstovui valdyti pavojingų šalių sąrašą?            &
		Reikalavimai, apibrėžiantys, jog sveikatos apsaugos ministerijos atstovas gali modifikuoti pavojingų šalių sąrašą                                                                               \\ \hline
		VKS-09                                                                                        &
		Ar gali E. policija sužinoti apie saviizoliacijos pažeidimus?                                 &
		Reikalavimai, apibrėžiantys, jog E. policijai pranešama apie asmens saviizoliacijos pažeidimą                                                                                                   \\ \hline
	\end{longtable}
\end{center}

\subsubsection{Pataisyta programinės įrangos reikalavimų specifikacijos versija}

\subsection{Programuotojo perspektyva}

Šiame poskyryje pateikiama programuotojo -- asmens, kursiančio aprašomą sistemą -- perspektyva.

\subsubsection{Programinės įrangos reikalavimų specifikacijų vertinimo kontrolinis sąrašas}

Šiame skirsnyje pateikiamas programinės įrangos reikalavimų specifikacijų vertinimo kontrolinis sąrašas,
paremtas programuotojo perspektyva.

\begin{center}
	\setstretch{1.0}
	\small
	\begin{longtable}{|C{2cm}|L{5cm}|L{8cm}|}
		\caption{Programuotojo kontrolinis sąrašas}
		\label{table:EmployeeSalary}
		\\ \hline
		\multicolumn{1}{|c|}{\makecell{\textbf{Kodas}}}                               &
		\multicolumn{1}{c|}{\makecell{\textbf{Klausimas/Teiginys}}}                   &
		\multicolumn{1}{c|}{\makecell{\textbf{Apibūdinimas}}}
		\\ \hline
		PKS-01                                                                        &
		Ar reikalavimas abstraktus?                                                   &
		Reikalavimas vadinamas abstrak2iu, jei jis suformuluotas remiantis juodos dėžės (angl. \textit{black box}) principu, t. y. apibrėžiama tik išoriškai matoma funkcinė ar nefunkcinė sistemos savybė ir nėrapateikiama jokių tos savybės realizavimo detalių.                                                                                                                                                                                   \\ \hline
		PKS-02                                                                        &
		Ar reikalavimas atominis?                                                     &
		Reikalavimas vadinamas atominiu, jei jis nebegali būti išskaidytas į smulkesnius. Reikalavimas turėtų būti sudarytas iš vieno atsekamo (angl \textit{traceable}) elemento.                                                                                                                                                                                                                                                                    \\ \hline
		PKS-03                                                                        &
		Ar reikalavimas nėra perteklinis?                                             &
		Reikalavimas vadinamas pertekliniu (angl. \textit{redundant}), jei jame kartojama informacija, pateikiama kituose reikalavimuose arba jei neįmanoma įvardinti, kokie veslo reikalavimai bus pasiekti įgyvendinus šį reikalavimą.                                                                                                                                                                                                              \\ \hline
		PKS-04                                                                        &
		Ar reikalavimas nėra dviprasmiškas?                                           &
		Reikalavimas turėtų būti interpretuojamas \textbf{vienareikšmiškai}. Tiesa, tai nereiškia, kad jis gali būti realizuojamas tik vienu būdu.                                                                                                                                                                                                                                                                                                    \\ \hline
		PKS-05                                                                        &
		Ar reikalavimas gali būti ištestuotas?                                        &
		Testuotojams turi būti įmanoma įvertinti, ar reikalavimas įgyvendintas teisingai: testo atsakymas turi būti arba teigiamas, arba neigiamas. Tam, jog galėtų būti ištestuojamas, reikalavimai turi būti aiškūs, tikslūs ir nedviprasmiški, o verifikacijos metodas -- realistiškas, t. y., jis turėtų nebūti per brangus, užimti per daug laiko ar reikalauti specifinių tikrintojo žinių ar specialios kompiuterinės bei programinės įrangos. \\ \hline
		PKS-06                                                                        &
		Ar reikalavimas pilnas?                                                       &
		Reikalavimas yra pilnas (angl. \textit{complete}), jei jame apibrėžiama viskas, kas turi būti apibrėžta ir jį perskaičius neišryškėja trūkumų (angl. \textit{issues}). Reikalavimas turi būti pilnas vertinant ne tik visumą, bet ir reikalavimą atskirai. Reikalavimas turi apimti visas įmanomas sąlygas.                                                                                                                                   \\ \hline
		PKS-07                                                                        &
		Ar reikalavimas tikslus?                                                      &
		Reikalavimas laikomas tiksliu, jei visi jame naudojami terminai yra apibrėžti ir nėra vartojama netikslių terminų (pavyzdžiui, beveik, apytiksliai, patogu, naudojama mažai atminties ir pan.).                                                                                                                                                                                                                                               \\ \hline
		PKS-08                                                                        &
		Ar reikalavimas aiškus, neperkrautas?                                         &
		Reikalavimas vadinamas neperkrautu, jei jame nėra argumentacijos, apibrėžimų ar kitos nereikalingos informacijos.                                                                                                                                                                                                                                                                                                                             \\ \hline
		PKS-09                                                                        &
		Ar reikalavimas parašytas visiems suinteresuotiems asmenims suprantama kalba? &
		Reikalavimas laikomas suprantamu, jei jame nėra specifinių terminų, neapibrėžtų terminų žodyne ir jame aiškiai apibrėžta, kokių funkcinių ar nefunkcinių charakteristikų sistema privalo turėti. Reikalavimai turi būti teisingi gramatiškai ir parašyti laikantis vieningo stiliaus. Turi būti paisoma standartinių konvencijų.                                                                                                              \\ \hline
	\end{longtable}
\end{center}

\subsubsection{Programinės įrangos sistemos reikalavimų specifikacijos įvertinimas}

\subsection{Operacijų ir palaikymo grupės perspektyva}

Šiame poskyryje pateikiama operacij7 ir palaikymo grupės -- asmenų, atsakingų už veikiančios
sistemos palaikymą: atnaujinimus ir kitas stabiliam sistemos veikimui reikalingas veiklas -- perspektyva.

\subsubsection{Programinės įrangos reikalavimų specifikacijų vertinimo kontrolinis sąrašas}

Šiame skirsnyje pateikiamas programinės įrangos reikalavimų specifikacijų vertinimo kontrolinis sąrašas,
paremtas operacijų ir palaikymo grupės perspektyva.

\begin{center}
	\setstretch{1.0}
	\small
	\begin{longtable}{|C{2cm}|L{5cm}|L{8cm}|}
		\caption{Operacijų ir palaikymo grupės kontrolinis sąrašas}
		\label{table:EmployeeSalary}
		\\ \hline
		\multicolumn{1}{|c|}{\makecell{\textbf{Kodas}}}                                                         &
		\multicolumn{1}{c|}{\makecell{\textbf{Klausimas/Teiginys}}}                                             &
		\multicolumn{1}{c|}{\makecell{\textbf{Apibūdinimas}}}
		\\ \hline
		OPKS-01                                                                                                 &
		Ar sistema gali būti atstatyta kritinio incidento atveju?                                               &
		Egzistuoja reikalavimai, apibūdinantys atsarginių kopijų darymą, bazinės linijos (angl. \textit{baseline}) konfigūracijos sudarymą ir priežiūrą, sistemos artefaktų saugojimą ir versijavimą. \\ \hline
		OPKS-02                                                                                                 &
		Ar sistema gali būti atnaujinta be veikimo sutrikimų (angl. \textit{downtime})?                         &
		Egzistuoja reikalavimai, apibrėžiantys sistemos pasiekiamumą (angl. \textit{availability}) bei reikalavimai, apibrėžiantys atnaujinimo procedūrą.                                             \\ \hline
		OPKS-03                                                                                                 &
		Ar sistemos žurnalai gali būti peržiūrėti?                                                              &
		Egzistuoja reikalavimai, apibrėžiantys sistemos įvykių bei klaidų žurnalizavimą bei būdus, kaip tuos žurnalus peržiūrėti                                                                      \\ \hline
		OPKS-04                                                                                                 &
		Ar sistemos veikimas gali būti stebimas?                                                                &
		Ar egzistuoja reikalavimai, apibrėžiantys sistemos naudojamų resursų bei apkrovos stebėjimą realiu laiku.                                                                                     \\ \hline
		OPKS-05                                                                                                 &
		Ar sistema pajėgi aptarnauti iki 1 milijono vartotojų (iš viso) ir 100 tūkstančių vartotojų vienu metu? &
		Ar sistema automatiškai plečiama išaugus užklausų skaičiui, ar numatytas didžiausias palaikomas bendras ir vienu metu sistema besinaudojančių vartotojų skaičius.                             \\ \hline
		OPKS-06                                                                                                 &
		Ar sistema atspari dažniausioms kibernetinėms atakoms?                                                  &
		Ar egzistuoja reikalavimai, apibrėžiantys sistemos atsparumą DDoS, SQL injekcijos ar kitoms OWASP top 10 sąraše apibūdinamoms atakoms.                                                        \\ \hline
		OPKS-07                                                                                                 &
		Ar pateikiama sistemos techninė specifikacija?                                                          &
		Ar apibrėžtas reikalavimas sistemos techninės specifikacijos sudarymui                                                                                                                        \\ \hline
		OPKS-08                                                                                                 &
		Ar egzistuoja galimybė įjungti ir išjungti tam tikras sistemos funkcijas?                               &
		Ar yra reikalavimas, apibrėžiantis tam tikrų sistemos funkcijų įjungimą/išjungimą atsiradus poreikiui (angl. \textit{feature flags})?                                                         \\ \hline
		OPKS-09                                                                                                 &
		Ar sudarytos galimybės spręsti paprastas vartotojų užklausas                                            &
		Ar yra galimybė inicijuoti vartotojo slaptažodžio keitimą, duomenų keitimą, profilio ištrynimą, peržiūrėti veiklą                                                                             \\ \hline
	\end{longtable}
\end{center}

\subsubsection{Programinės įrangos sistemos reikalavimų specifikacijos įvertinimas}

\subsection{Pataisyta programinės įrangos reikalavimų specifikacijos versija}
\begin{center}
	\setstretch{1.0}
	\small
	\begin{longtable}{|C{2cm}|L{5cm}|L{8cm}|}
		\caption{Operacijų ir palaikymo grupės kontrolinis sąrašas}
		\label{table:EmployeeSalary}
		\\ \hline
		\multicolumn{1}{|c|}{\makecell{\textbf{Kodas}}}                                 &
		\multicolumn{1}{c|}{\makecell{\textbf{Reikalavimas}}}                     &
		\multicolumn{1}{c|}{\makecell{\textbf{Apibūdinimas}}}
		\\ \hline
		\reqCode &
		Galimybė valdyti užsikrėtusio žmogaus duomenis&
		{\color{blue} 1.Privatus asmuo užfiksuoja savo saviizoliaciją.} \textbf{Pradiniai duomenys} - žmogaus asmeniniai duomenys, kontaktiniai duomenys, saviizoliacijos pradžios ir numatomos pabaigos data, sarašas su kuo kontaktavo susirgęs žmogus.
		\textbf{Gautas rezultatas} - įvesti žmogaus saviizoliacijos duomenis į registrą, galimybė tos duomenis perduoti kitoms posistemėms. {\color{blue} 2. Privačiam asmeniui nustatoma saviizoliacijos pradžia ir pabaiga. \textbf{Pradiniai duomenys} - užsikrėtusių žmonių registro duomenys, sąrašas su kuo kontaktavo susirgęs žmogus. \textbf{Gautas rezultatas} - žinutė, informuojanti asmenį apie privalomą informaciją. Pateikiama saviizoliacijos pradžios ir numatomos pabaigos datos. 3. Užsikrėtęs privatus asmuo gali užfiksuoti asmenis, su kuriais kontaktavo \textbf{Pradiniai duomenys} - kontaktuotų žmonių asmeniniai duomenys, kontaktiniai duomenys. \textbf{Gautas rezultatas} - įvesti kontaktuotų žmonių duomenys į registrą, galimybė tuos duomenis perduoti kitoms posistemėms.}\\ \hline
		\reqCode &
		Galimybė valdyti pavojingų šalių registrą&
		{\color{blue} 1.Sveikatos apsaugos ministerijos atstovas gali papildyti pavojingų šalių sąrašą }\textbf{Pradiniai duomenys} - nauja informaciją apie pavojingas šalis. \textbf{Gautas rezultatas} - atnaujintas pavojingų šalių sąrašas kurį naudos kitos posistemės.
		{\color{blue} 2.Sveikatos apsaugos ministerijos atstovas gali pašalinti šalį iš pavojingų šalių sąrašo. \textbf{Pradiniai duomenys} - šalies pavadinimas, esantis pavojingų šalių sąraše. \textbf{Gautas rezultatas} - atnaujintas pavojingų šalių sąrašas, kurį naudos kitos posistemės.}\\ \hline	
		\reqCode &
		Galimybė valdyti saviizoliacijos pranešimų sistema&
		{\color{blue} 1. Privatus asmuo gauną pranešimą apie privalomą saviizoliaciją. \textbf{Pradiniai duomenys} - gautas užsikrėtusio asmens kontaktuotų asmenų sąrašas. \textbf{Gautas rezultatas} - žinutėm pranešanti apie asmens privalomą saviizoliaciją, saviizoliacijos pradžios ir numatomos pabaigos datos.}\textbf{Pradiniai duomenys} - užsikrėtusių žmonių registro duomenys, žmogaus lokacijos duomenys. \textbf{Gautas rezultatas} - žmogus informuotas apie jam paskitrą saviizoliacijos laikotarpį SMS žinute. Policija informuojama žmogui pažeidus saviizoliaciją.
		{\color{blue} 2. E.policia gauna pranešimus apie asmens saviizoliacijos pažeidimą. \textbf{Pradiniai duomenys} - užsikrėtusio asmens asmeniniai duomenys, saviizoliacijos praždios ir pabaigos datos, GPS lokacija. \textbf{Gautas rezultatas} - E.policijai gaunama žinutė, nusakanti, koks asmuo pažeidė saviizoliaciją, jų asmeniniai duomenys ir GPS lokacija}\\ \hline	
		\reqCode &
		Galimybė valdyti portalo apie saviizoliacijos informaciją&
		\textbf{Pradiniai duomenys} - užsikrėtusiųjų registras, žmogaus lokacijos duomenys. \textbf{Gautas rezultatas} - gauti ir atvaizduojama žmogaus saviizoliacijos būsena.\\ \hline		
		\reqCode &
		Galimybė valdyti mobilioji aplikaciją saviizoliacijai sekti&
		\textbf{Pradiniai duomenys} - žmogaus GPS lokacija, užsikėtusiųjų registras. \textbf{Gautas rezultatas} - įsitikint, kad žmogus nepažeidžia saviizoliacijos reikalavimų.\\ \hline								\reqCode &
		Programų sistemos sąveikos su kitomis sistemomis&
		Sistema sąveikaus su: e. policijos sistema, NVSC sistema, sveikatos ministerijos sistema, muitinės sistema, e. sveikatos sistema, karštosios linijos sistema, renginų organizatorių sistema.\\ \hline		
		\reqCode &
		Programų sistemos atitikimas galiojantiems teisės aktams&
		Sistema atitinka Lietuvos ir Europos teisės aktams. 
		Sistema taip pat laikosi BDAR reglamento.\\ \hline	
		\reqCode &
		Programų sistemos trasuojamumo reikalavimai&
		Sistemos atrasuojamumas bus įgyvendinatas pasinaudojant Jira Atlassian sistema kuri užtikrina programos trsuojamumą\\ \hline	
		\reqCode &
		Programų sistemos patikimumo reikalavimai&
		Sistema turi išlikti pasiekiama 99 procentus laiko.
		Įvykus sistemos tiktims sistema turi sugebėti automatiškai pasileisti išnaujo neprarasdama duomenų.\\ \hline
		\reqCode &
		Programų sistemos išbaigtumas&
		Užsakovui priimtimas vienas tikdis per dieną.
		Trumpiausias laikas tarp dviejų trikdžių yra 24 valandos.
		Sistemoje neturi būti palikta esminių klaidų.\\ \hline	
		\reqCode &
		Programų sistemos atsparumo triktims reikalavimai&
		Dėl trikdžiū sistema gali neveikti 1 valandą per diena.
		Sistemoje gali būti vienas įsilaužimas per mėsenį.\\ \hline	
		\reqCode &
		Programų sistemos atkuriamumo reikalavimai&
		Sistema turi galėti atsikurti per valandą po sutrikimo.
		Per valandą turi būti atkūriami prarasti duomenys ir funkcionalumas.
		Trikdis turi būti rastas ir pašalintas per savaitę.		\\ \hline	
		\reqCode &
		Programų sistemos prieinamumo reikalavimai&
		Per dieną sistema turi išlikti funkcionali 23 valandas. 		\\ \hline	
		\reqCode &
		Programų sistemos pažeidžiamumo reikalavimai&
		Esminis sistemos funkcionalumas turi būti atkurtas per valandą.\\ \hline	
		\reqCode &
		Programų sistemos aptarnavimo reikalavimai&
		Sistema turi būti galima atstatyi per valandą.
		Surastas trigdis programiniam kode turi būti pašalintas per savaitę.		\\ \hline	
		\reqCode &
		Programų sistemos diegimo reikalavimai&
		Sistema yra paleidžiama debesų kompiuterijos srveriuose, todėl sistemos diegimo kaštai yra minimlūs.
		Sistemai perkelti iš vieno debesijos serverio į kitą turi pakakti 24 valandų darbo.
		Sistema perkeliama perkelus sukompiliuotus binarinius failus.		\\ \hline	
		\reqCode &
		Programų sistemos adaptuojamumo reikalavimai&
		Perkelti sistema ant naujos platformos kainuotų 10000 žmogaus darbo valandų.
		Perkelti sistemą į kitą kompiuterinę platformą 20 žmogaus darbo valandų
		Perkelti sistemą į naujasnę operacinės sistemos versiją kainuotų 100 žmogaus darbo valandų
		Perkelti sistemos duomenis į nauja DBVS kainuotų 200 žmogaus darbo valandų.\\ \hline	
		\reqCode &
		Programų sistemos instaliuojamumo reikalavimai&
		Programai instaliuoti turi pakakti 40 žmogaus darbo valandų. 
		Maksimalus procentas kodo kurį reiktų pakeisti instaliuojant sistemą yra 0,01\%.
		Maksimalus procentas failų kuriuos reiktų pakeisti instaliuojant sistemą yra 0,01\%.\\ \hline		
		\reqCode &
		Programų sistemos atitikimo keliamumo standartams reikalavimai&
		Keliamumo reikalavimai atitinką apibrėžtą vidinį standartą pagal ISO.\\ \hline	
		\reqCode &
		Programų sistemos pakeičiamumo reikalavimai&
		Sistema nauja, todėl nėra senos sistemos kurią būtų galima pakeisti.\\ \hline									\reqCode &
		Programų sistemos analizuojamumo reikalavimai&
		Trigdžio priežastis sistemoje turi būti išsiaiškinta per 20 žmogaus darbo valandų\\ \hline	
		\reqCode &
		Programų sistemos keičiamumo reikalavimai&
		Trigžio priežastis sistemoje turi būti pašalinta per 20 žmogaus darbo valandų.\\ \hline		
		\reqCode &
		Programų sistemos stabilumo reikalavimai&
		Mažiausias vidutinis laikas kurį sistema bus įtakota dėl atsitiktinio neteisingo pakeitimo yra 4 valandos.\\ \hline		
		\reqCode &
		Programų sistemos testuojamumo reikalavimai&
		Testai sistemoje nevykdomi.\\ \hline	
		\reqCode &
		Programų sistemos tvarkomumo reikalavimai&
		Sistema turi būti atkuriama per 1 žmogaus darbo valandą.\\ \hline	
		\reqCode &
		Programų sistemos pakartotino panaudojamumo reikalavimai&
		Sistema negali būti perpanaudota.\\ \hline	
		\reqCode &
		\color{blue}Vartotojo sąsajoje esančio teksto kalbos reikalavimas&
		\color{blue}Vartotojo sąsajoje esantis tekstas bus pateikiamas lietuvių kalba.\\ \hline										\reqCode &
		\color{blue}Vartotojo sąsajos interaktyvumo reikalavimai&
		\color{blue} Sistemos vartotojo sąsaja reaguos į vartotojo interakcijas - vartotojui įvykdžius sistemoje veiksmą, sistema atvaizduos atitinkamą atoveiksmį: pranešimo žinutę, naujo turinio pavaizdavimą, mygtuko spalvos pakitimas jį paspaudžius ir kt.\\ \hline																	
	\end{longtable}
\end{center}
		
\section{Reikalavimų nuleidimas žemyn - kokybės namas}
\subsection{Klientas}
\subsubsection{Produkto (sistemos) planavimas}
\subsubsection{Komponentų diegimas}
\subsubsection{Komponentų diegimas}
\subsubsection{Gamybos planavimas}
\subsection{Vartotojas}
\subsubsection{Produkto (sistemos) planavimas}
\subsubsection{Komponentų diegimas}
\subsubsection{Komponentų diegimas}
\subsubsection{Gamybos planavimas}
\subsection{Programuotojas}
\subsubsection{Produkto (sistemos) planavimas}
\subsubsection{Komponentų diegimas}
\subsubsection{Komponentų diegimas}
\subsubsection{Gamybos planavimas}
\subsection{Operacijų ir palaikymo grupė}
\subsubsection{Produkto (sistemos) planavimas}
\subsubsection{Komponentų diegimas}
\subsubsection{Komponentų diegimas}
\subsubsection{Gamybos planavimas}

\section{Išvada}

\end{document}

