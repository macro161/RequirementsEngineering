\documentclass{VUMIFPSkursinis}
\usepackage{algorithmicx}
\usepackage{algorithm}
\usepackage{algpseudocode}
\usepackage{amsfonts}
\usepackage{amsmath}
\usepackage{bm}
\usepackage{caption}
\usepackage{color}
\usepackage{float}
\usepackage{graphicx}
\usepackage{listings}
\usepackage{subfig}
\usepackage{array}
\usepackage{wrapfig}
\usepackage{tabu}

% Titulinio aprašas
\university{Vilniaus universitetas}
\faculty{Matematikos ir informatikos fakultetas}
\department{}
\papertype{Reikalavimų inžinerijos laboratorinis darbas}
\title{Reikalavimų specifikacija}
\titleineng{Requirement specification}
\status{1 kurso magistratūros studentai}
\author{Šarūnas Kazinieras Buteikis, Matas Savickis}
\secondauthor{Rokas Ulicas, Vytautas Krivickas}
\supervisor{dr. Audronė Lupeikienė}
\date{Vilnius – \the\year}

% Nustatymai
% \setmainfont{Palemonas} % Pakeisti teksto šriftą į Palemonas (turi būti įdiegtas sistemoje)
\bibliography{bibliografija}

\begin{document}

\maketitle

\tableofcontents

\section{Reikalavimų artefaktai}

\begin{table}[h!]
\begin{tabular}{|p{4cm}|p{4cm}|p{4cm}|p{4cm}|}
\hline
                                                                                                              & Kodėl? (motyvacija) & \multicolumn{1}{c|}{Kaip? (veiklos)} & Ką? (apdorojami objektai) \\ \hline
\begin{tabular}[c]{@{}l@{}}Veikslo reikalavimai \\ (verslo inžinieriaus \\ požiūris)\end{tabular}             &                     &\begin{enumerate}
																																												\item{Tvarkyti visą informaciją susijusią su žmonėmis 
																																							sergančiais koronos virusu ir informuoti pacientus 
																																							apie jų seikatos būklę bei galimas virusinio susirgimo rizikas.}
																																												\item{Galimybė tvarkyti pacientų užsikrėtimų įrašus.}
																																												\item{Galimybė pacientui paskirti viruso tyrimą.}
																																												\item{Galimybė pacientui paskirti viruso antikūnių tyrimą.}
																																												\item{Galimybė pacientui pranešti apie viruso tyrimo rezultatus.}																																												\item{Galimybė registruoti paciento buvimo vietas.}
																																												\item{Galimybė pacientui praneši apie viruso židinius.}
																																												\item{Galimybė pacientui paskirti vakciną.}
																																												\item{Galimybė pacientui paskirti gydymą.}
																																												\item{Galimybė registruoti paciento gydymo eigą.}


																																											\end{enumerate}
&                           \\ \hline
\begin{tabular}[c]{@{}l@{}}Vartotojo reikalavimai \\ (dalykinės srities \\ specialisto požiūris)\end{tabular} &                     &                                      &                           \\ \hline
\begin{tabular}[c]{@{}l@{}}IS reikalavimai \\ (IS inžinieriaus \\ požiūris)\end{tabular}                      &                     &                                      &                           \\ \hline
\begin{tabular}[c]{@{}l@{}}PS reikalavimai \\ (sisteminio analitiko \\ požiūris)\end{tabular}                 &                     &                                      &                           \\ \hline
\begin{tabular}[c]{@{}l@{}}Projektiniai PS \\ reikalavimai \\ (PS inžinieriaus \\ požiūris)\end{tabular}      &                     &                                      &                           \\ \hline
\begin{tabular}[c]{@{}l@{}}Realizaciniai PS \\ reikalavimai \\ (programuotojo požiūris)\end{tabular}          &                     &                                      &                           \\ \hline
\end{tabular}
\end{table}



\begin{table}[h!]
\begin{tabular}{|l|l|l|l|}
\hline
                                                                                                              & Kas? (funkciniai vienetai) & \multicolumn{1}{c|}{Kur? (vieta)} & Kada? (laikas) \\ \hline
\begin{tabular}[c]{@{}l@{}}Veikslo reikalavimai \\ (verslo inžinieriaus \\ požiūris)\end{tabular}             &                            &                                   &                \\ \hline
\begin{tabular}[c]{@{}l@{}}Vartotojo reikalavimai \\ (dalykinės srities \\ specialisto požiūris)\end{tabular} &                            &                                   &                \\ \hline
\begin{tabular}[c]{@{}l@{}}IS reikalavimai \\ (IS inžinieriaus \\ požiūris)\end{tabular}                      &                            &                                   &                \\ \hline
\begin{tabular}[c]{@{}l@{}}PS reikalavimai \\ (sisteminio analitiko \\ požiūris)\end{tabular}                 &                            &                                   &                \\ \hline
\begin{tabular}[c]{@{}l@{}}Projektiniai PS \\ reikalavimai \\ (PS inžinieriaus \\ požiūris)\end{tabular}      &                            &                                   &                \\ \hline
\begin{tabular}[c]{@{}l@{}}Realizaciniai PS \\ reikalavimai \\ (programuotojo požiūris)\end{tabular}          &                            &                                   &                \\ \hline
\end{tabular}
\end{table}


\section{Verslo reikalavimai}
	
	\subsection{Kodėl?}

		\subsubsection{Išorinė verslo analizė}

		\subsubsection{Vidinė verslo analizė}

		\subsubsection{SWOT analizė}

		\subsubsection{Siūloma verslo strategija}

		\subsubsection{Tikslų medis}

	\subsection{Kaip?}

	\subsection{Kas?}

	\subsection{Kieno?}

	\subsection{Kur?}

	\subsection{Kada?}

\end{document}

